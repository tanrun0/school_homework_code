% !TEX program = xelatex
\documentclass[twocolumn]{article}
\usepackage{ctex}
\usepackage{amsmath}
\usepackage{graphicx}
\usepackage{xcolor}
\usepackage{listings}
\usepackage{geometry}
\usepackage{fancyhdr}
\usepackage{tabularx}
\usepackage{booktabs}
\usepackage{hyperref}

% --- 页面布局 ---
\geometry{a4paper, left=2cm, right=2cm, top=2.5cm, bottom=2.5cm, columnsep=1cm}

% --- 代码高亮配置 ---
\definecolor{codegreen}{rgb}{0,0.6,0}
\definecolor{codegray}{rgb}{0.5,0.5,0.5}
\definecolor{codepurple}{rgb}{0.58,0,0.82}
\definecolor{backcolour}{rgb}{0.95,0.95,0.92}

\lstdefinestyle{mystyle}{
    backgroundcolor=\color{backcolour},   
    commentstyle=\color{codegreen},
    keywordstyle=\color{magenta},
    numberstyle=\tiny\color{codegray},
    stringstyle=\color{codepurple},
    basicstyle=\ttfamily\footnotesize,
    breakatwhitespace=false,         
    breaklines=true,                 
    captionpos=b,                    
    keepspaces=true,                 
    numbers=left,                    
    numbersep=5pt,                  
    showspaces=false,                
    showstringspaces=false,
    showtabs=false,                  
    tabsize=2,
    columns=flexible,
    literate={\_}{{\textunderscore}}1,
    mathescape=false,
    texcl=false
}
\lstset{style=mystyle}

% --- 页眉页脚 ---
\pagestyle{fancy}
\fancyhf{}
\fancyhead[L]{COOL Parser Lab Report}
\fancyhead[R]{\thepage}
\fancyfoot[C]{\small \copyright\ 2025}
\renewcommand{\headrulewidth}{0.4pt}
\renewcommand{\footrulewidth}{0.4pt}

% --- 标题 ---
\title{
    \vspace{-1cm} % 调整标题位置
    \textbf{COOL 语法解析器(Parser)实验报告} \\
    \large \texttt{Compiler Principle Assignment}
}
\author{
    姓名: \textbf{谭润} \\
    学号: \textbf{20238131027} \\
    班级: \textbf{大数据二班} \\
    项目代码:\href{https://github.com/tanrun0/school_homework_code/tree/main/compiler/shiyan3}{Github链接}
}
\date{\today}

% --- 文档开始 ---
\begin{document}

\maketitle
\thispagestyle{fancy} % 首页也使用fancy样式

\begin{abstract}
\noindent
本文基于简化版 COOL 语法完成了一个功能齐全的解析器示例,配合手写的 Flex 词法分析器用于测试语法功能。实验中实现了类和继承、属性、方法、表达式(函数调用、分支、循环、let、case)、常量和基础运算符的解析功能,并输出简单的树状结构以验证解析正确性。实验使用 Bison(Yacc)和 Flex 构建解析管道,并提供 Makefile 便于快速构建和运行测试用例。

\end{abstract}

\section{任务与设计目标}

本次实验目标是实现 COOL 语言的一个基础解析器(Parser),能够将词法分析器(Lexer)生成的 token 流解析为树状结构(AST)。主要目标包括:

\begin{itemize}
    \item 实现类/继承、方法/属性解析
    \item 处理表达式(if/while/let/dispatch、操作符)
    \item 能够解析并报告语法错误,并在测试文件(good.cl / bad.cl / stack.cl / complex.cl)上演示
\end{itemize}

\section{实现说明}

\subsection{代码结构}
\begin{verbatim}
My_work/
|-- cool.y        
|-- lexer.l          
|-- parser.y       
|-- Makefile       
|-- README.md      
|-- report_final.tex 
|-- run_tests.ps1    
\end{verbatim}



\subsection{核心实现要点}

\begin{itemize}
    \item Lexer 使用 Flex 编写,实现关键词大小写不敏感识别,支持 `TYPEID`/`OBJECTID` 区分,字符串转义解析、嵌套注释与错误检测。
    \item Parser 使用 Bison 编写,在语义动作中调用提供的 `cool-tree.h` 的构造函数以生成 AST,并使用 `SET\_NODELOC`/`@\$` 来记录位置。
    \item 为便于演示,实现了一个简短的 AST 打印器 "print\_node",可以在解析成功时打印解析树。
    \item 本项目参考了 `cool.y`,实现中保留了要求的语法规则和 AST 构建约定,并在语义动作中使用 `SET\_NODELOC` 为 AST 设置行号,便于后续语义分析和错误报告。
\end{itemize}

\subsection{主要实现}
\begin{itemize}
    \item feature\_list: 支持空列表与带分号的 feature 列表,解决部分边界导致解析失败的问题。
    \item let 表达式: 实现了多变量 let 的嵌套解析,将多个绑定展开成嵌套的 let 调用,保持语义正确。
    \item Node 位置定位: 在每个语义动作中使用 `@\$` 和 `SET\_NODELOC(@1)` 显式设置位置,使后续错误报告更准确。
    \item 错误恢复: 在一些规则中使用 `error` 产生式并创建默认 AST 节点,以便解析器在错误时继续处理后续输入。
\end{itemize}

\section{构建与运行}
\subsection{依赖}
\begin{itemize}
    \item flex (>= 2.6.x)
    \item bison (>= 3.x)
    \item g++ (支持 C++11 或更高)
    \item Make(建议用于快速构建)
    \item 推荐在 WSL 或 Linux 环境中执行(Windows 环境下需安装 MSYS/Cygwin 或 WSL 来运行 `bison`/`flex`/`make`)
\end{itemize}

\subsection{开发环境}
\begin{itemize}
    \item 操作系统:Ubuntu 22.04.1 LTS
    \item 内核版本:5.15.0-60-generic
    \item Flex 版本:2.6.4
    \item G++ 版本:g++ (Ubuntu 11.3.0) 11.3.0
    \item Make 版本:GNU Make 4.3
    \item SPIM 版本:SPIM Version 8.0
\end{itemize}

\subsection{构建命令}
\begin{verbatim}
cd d:\\VS_code\\compile\\shiyan3\My_work
make

或者手动:
bison -d -o cool-parse.cc cool_ai.y
flex -o lex.yy.c lexer.l
g++ -std=c++11 -Wall -g cool-parse.cc lex.yy.c
-lfl -o parser
\end{verbatim}

\subsection{运行示例}
\begin{verbatim}
运行示例(使用 demo parser/lexer):
./parser ..\good.cl
./parser ..\bad.cl
./parser ..\stack.cl
\end{verbatim}

\section{测试说明与结果}

\subsection{基础测试}
使用提供的 `good.cl`进行测试,构建并运行解析器得到解析成功,并打印树结构用于验证。

示例输出(节选,运行 `./parser ..\\good.cl`):
\begin{lstlisting}
Parse succeeded
Classes
    Class
        type: A
        Features
            Method
                name: ana
                // feature: ana
\end{lstlisting}

\subsection{错误测试}
对 `bad.cl` 进行测试,解析器会检测语法问题(例如继承关键字拼写错误、缺少大括号等),输出示例(运行 `./parser ..\\bad.cl`):
\begin{lstlisting}
Parse error at line 8: syntax error at or near TYPEID
Parse failed
\end{lstlisting}

\section{遇到的问题与解决方案}
在实验过程中,主要遇到以下挑战:
\begin{itemize}
    \item 字符串与注释边界处理:由于字符串需支持转义字符并且不能跨行,采用独占状态(`\%x`)并在字符串状态内解析转义字符,遇到换行或 EOF 则返回适当的 ERROR Token;注释实现支持多行和嵌套,采用计数器和注释状态管理。
    \item 工具环境限制(Windows):在默认 Windows PowerShell 中通常缺少 `bison`/`flex`/`make`,建议在 WSL/Ubuntu 环境下运行编译测试;在 Windows 环境下也可使用 MSYS2/Cygwin 或安装相关工具后再运行。
    \item 与课程 AST 集成:将 Bison 的语义动作映射到 `cool-tree.h` 中的构造函数需要仔细对齐类型签名和节点位置语义(`SET\_NODELOC`),否则编译或链接阶段会报错;我在 `cool.y` 中遵循标准接口以保证兼容性。
    \item let 等复杂语法:多变量 let 的嵌套需要在语义动作中通过嵌套 `let` 调用来还原 AST 结构,若实现不当会导致语法和语义不一致。
\end{itemize}

\section{总结}
本实验实现了一个简化但功能完整的 COOL 语法解析器,支持核心语法结构(如类、继承、属性、方法、表达式、分发调用、let、case 等)的解析。实现的 `cool\_ai.y`(位于 `My\_work/`)已与课程 AST 构造函数对接,并做了错误恢复处理;`lexer.l` 与 `parser.y` 提供了用于快速测试的演示实现。使用 Flex(词法)和 Bison(语法)提供方便的解析开发与测试流程。该实现为课程作业提交提供了完整的 `cool.y` 参考实现,并可在必要时进一步对齐课程评分工具链。

\section{实现范围与局限}
本次实现覆盖了课程要求的主要解析规则与语义动作,详细实现如下:
\begin{itemize}
    \item 已实现:类、继承、方法、属性、常见表达式(算术、比较、赋值、if、while、case、let、dispatch 等)、常量、字符串与注释处理、错误检测与恢复。
    \item 局限:完全的类型检查、语义分析、代码生成(Semant/Cgen)未在本项目中实现;另外对字符串极端边界(极长字符串)额外校验策略可进一步加强。
\end{itemize}


\appendix
\section{附录:cool.y 核心源码片段}
\begin{lstlisting}[language=C]
%{
#include "cool-tree.h"
#include "stringtab.h"
#include "utilities.h"
#include <iostream>
using namespace std;

extern char *curr_filename;
extern int curr_lineno;

// 位置处理
#define YYLTYPE int
extern int node_lineno;
#define SET_NODELOC(Loc) node_lineno = Loc;

void yyerror(const char *s);
extern int cool_yylex();
static int yylex_wrapper() {
    extern YYLTYPE cool_yylloc;
    int tok = cool_yylex();
    cool_yylloc = curr_lineno;
    return tok;
}
#define yylex yylex_wrapper

Program ast_root;
Classes parse_results;
int omerrs = 0;
%}

// 语法单元类型定义
%union {
  Boolean boolean;
  Symbol symbol;
  Program program;
  Class_ class_;
  Classes classes;
  Feature feature;
  Features features;
  Formal formal;
  Formals formals;
  Case case_;
  Cases cases;
  Expression expression;
  Expressions expressions;
  char *error_msg;
}

// Token定义
%token CLASS ELSE FI IF IN
%token INHERITS LET LOOP POOL THEN WHILE
%token CASE ESAC OF DARROW NEW ISVOID NOT
%token <symbol> STR_CONST INT_CONST BOOL_CONST TYPEID OBJECTID
%token ASSIGN LE ERROR

// 运算符优先级与结合性
%nonassoc IN
%right ASSIGN NOT
%nonassoc LE '<' '='
%left '+' '-'
%left '*' '/'
%left ISVOID
%left '~' '@' '.'

// 非终结符类型声明
%type <program> program
%type <classes> class_list
%type <class_> class
%type <features> feature_list
%type <feature> feature
%type <formals> formal_list
%type <formal> formal
%type <cases> case_list
%type <case_> case_branch
%type <expressions> expr_list expr_block_list
%type <expression> expr

%start program
%%

// 程序入口
program:
  class_list { @$ = @1; ast_root = program($1); }
  ;

// 类列表
class_list:
  class { @$ = @1; $$ = single_Classes($1); }
  | class_list class { @$ = @1; $$ = append_Classes($1, single_Classes($2)); }
  ;

// 类定义(含继承)
class:
  CLASS TYPEID '{' feature_list '}' ';' {
      @$ = @1;
      $$ = class_($2, idtable.add_string("Object"), $4, stringtable.add_string(curr_filename));
  }
  | CLASS TYPEID INHERITS TYPEID '{' feature_list '}' ';' {
      @$ = @1;
      $$ = class_($2, $4, $6, stringtable.add_string(curr_filename));
  }
  | error ';' {
      @$ = @1;
      $$ = class_(idtable.add_string("Error"), idtable.add_string("Object"), nil_Features(), stringtable.add_string(curr_filename));
  }
  ;

// 特征列表(方法/属性)
feature_list:
  /* 空列表 */ { $$ = nil_Features(); }
  | feature_list feature ';' { @$ = @2; $$ = append_Features($1, single_Features($2)); }
  ;

// 特征(方法/属性)
feature:
  // 方法定义
  OBJECTID '(' formal_list ')' ':' TYPEID '{' expr '}' {
      @$ = @1; SET_NODELOC(@1); $$ = method($1, $3, $6, $8);
  }
  // 无初始值的属性
  | OBJECTID ':' TYPEID {
      @$ = @1; node_lineno = 0; Expression ne = no_expr(); SET_NODELOC(@1); $$ = attr($1, $3, ne);
  }
  // 有初始值的属性
  | OBJECTID ':' TYPEID ASSIGN expr {
      @$ = @1; SET_NODELOC(@1); $$ = attr($1, $3, $5);
  }
  ;

// 形参列表
formal_list:
  /* 空列表 */ { $$ = nil_Formals(); }
  | formal { $$ = single_Formals($1); }
  | formal_list ',' formal { @$ = @1; $$ = append_Formals($1, single_Formals($3)); }
  ;

// 单个形参
formal:
  OBJECTID ':' TYPEID { @$ = @1; $$ = formal($1, $3); }
  ;

// 表达式块(分号分隔)
expr_block_list:
  expr ';' { $$ = single_Expressions($1); }
  | expr_block_list expr ';' { @$ = @1; $$ = append_Expressions($1, single_Expressions($2)); }
  ;

// 核心表达式(含赋值、调用、分支、循环等)
expr:
  // 变量赋值
  OBJECTID ASSIGN expr { @$ = @1; SET_NODELOC(@1); $$ = assign($1, $3); }
  // 静态方法调用
  | expr '@' TYPEID '.' OBJECTID '(' expr_list ')' { @$ = @1; SET_NODELOC(@1); $$ = static_dispatch($1, $3, $5, $7); }
  // 动态方法调用
  | expr '.' OBJECTID '(' expr_list ')' { @$ = @1; SET_NODELOC(@1); $$ = dispatch($1, $3, $5); }
  // 无接收者的方法调用(默认self)
  | OBJECTID '(' expr_list ')' { @$ = @1; SET_NODELOC(@1); $$ = dispatch(object(self_sym), $1, $3); }
  // 条件分支
  | IF expr THEN expr ELSE expr FI { @$ = @1; SET_NODELOC(@1); $$ = cond($2, $4, $6); }
  // 循环
  | WHILE expr LOOP expr POOL { @$ = @1; SET_NODELOC(@1); $$ = loop($2, $4); }
  // 模式匹配
  | CASE expr OF case_list ESAC { @$ = @1; SET_NODELOC(@1); $$ = typcase($2, $4); }
  // 代码块
  | '{' expr_block_list '}' { @$ = @1; SET_NODELOC(@1); $$ = block($2); }
  // 单变量let表达式
  | LET OBJECTID ':' TYPEID IN expr { @$ = @1; SET_NODELOC(@1); node_lineno = 0; Expression ne = no_expr(); SET_NODELOC(@1); $$ = let($2, $4, ne, $6); }
  // 带初始值的单变量let
  | LET OBJECTID ':' TYPEID ASSIGN expr IN expr { @$ = @1; SET_NODELOC(@1); $$ = let($2, $4, $6, $8); }
  // 多变量嵌套let
  | LET OBJECTID ':' TYPEID ',' OBJECTID ':' TYPEID IN expr { @$ = @1; SET_NODELOC(@1); node_lineno = 0; Expression e1 = no_expr(); SET_NODELOC(@1); node_lineno = 0; Expression e2 = no_expr(); $$ = let($2, $4, e1, let($6, $8, e2, $10)); }
  // 算术运算
  | expr '+' expr { @$ = @1; SET_NODELOC(@1); $$ = plus($1, $3); }
  | expr '-' expr { @$ = @1; SET_NODELOC(@1); $$ = sub($1, $3); }
  | expr '*' expr { @$ = @1; SET_NODELOC(@1); $$ = mul($1, $3); }
  | expr '/' expr { @$ = @1; SET_NODELOC(@1); $$ = divide($1, $3); }
  // 一元运算
  | '~' expr { @$ = @1; SET_NODELOC(@1); $$ = neg($2); }
  | NOT expr { @$ = @1; SET_NODELOC(@1); $$ = comp($2); }
  | ISVOID expr { @$ = @1; SET_NODELOC(@1); $$ = isvoid($2); }
  // 比较运算
  | expr '<' expr { @$ = @1; SET_NODELOC(@1); $$ = lt($1, $3); }
  | expr '=' expr { @$ = @1; SET_NODELOC(@1); $$ = eq($1, $3); }
  | expr LE expr { @$ = @1; SET_NODELOC(@1); $$ = leq($1, $3); }
  // 新建对象
  | NEW TYPEID { @$ = @1; SET_NODELOC(@1); $$ = new_($2); }
  // 常量
  | INT_CONST { @$ = @1; SET_NODELOC(@1); $$ = int_const($1); }
  | BOOL_CONST { @$ = @1; SET_NODELOC(@1); $$ = bool_const($1); }
  | STR_CONST { @$ = @1; SET_NODELOC(@1); $$ = string_const($1); }
  // 变量引用
  | OBJECTID { @$ = @1; SET_NODELOC(@1); $$ = object($1); }
  // 括号表达式
  | '(' expr ')' { @$ = @1; $$ = $2; }
  ;

// case分支列表
case_list:
  case_branch { $$ = single_Cases($1); }
  | case_list case_branch { @$ = @1; $$ = append_Cases($1, single_Cases($2)); }
  ;

// 单个case分支
case_branch:
  OBJECTID ':' TYPEID DARROW expr ';' { @$ = @1; SET_NODELOC(@1); $$ = branch($1, $3, $5); }
  ;

%%

// 全局变量初始化
int curr_lineno = 1;
Symbol self_sym = idtable.add_string("self");

// 语法错误处理
void yyerror(const char *s) {
  extern char *curr_filename;
  cerr << '"' << curr_filename << "", line " << curr_lineno << ": " << s << " at or near ";
  print_cool_token(yychar);
  cerr << endl;
  omerrs++;
  if (omerrs > 50) {
    fprintf(stdout, "More than 50 errors\n");
    exit(1);
  }
}
\end{lstlisting}

\end{document}
